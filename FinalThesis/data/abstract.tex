% !TeX root = ../thuthesis-example.tex

% 中英文摘要和关键字

\begin{abstract}
随着科技的进步,汽车行业的快速发展和广泛应用,车辆之间的交互将变得必不可少。为了创建一个更加安全高效的交通环境,区块链技术已经被应用于车联网中。区块链的去中心化、独立性等特点使得在出租车调度系统中应用区块链技术能够消除中介,允许乘客与司机的直接交流与交易,在网络出租车服务中使用区块链有助于参与的所有利益相关方关系更加紧密。

然而传统的区块链是单链结构,在处理大量交易时,会导致链条过长,从而消耗大量时间和能量,进而执行效率低下。为解决该问题,我们在出租车调度系统中应用“树状区块链”。树状区块链根据地理位置的Geohash编码划分子链,形成分支区块管理其下子链,不同子链管理不同区域的这样一种树状字典树结构。

本文首先是阅读了树状区块链的部分实现源码,对于一些源码核心功能函数给出了注释说明,之后,基于现有项目结构完成了对树状区块链应用于出租车调度系统的复现工作,并验证多子链并行运行的负载性能。此外,本文还验证了树状区块链的跨子链资产转移功能的正确性,并测试了其性能。在以上工作的基础之上,本文尝试对现有出租车调度系统的跨区域(跨子链)交易进行改进。现有的出租车调度系统仅支持一个子链内部的交易,但在实际应用场景中,乘客与车辆分处不同区域(不同子链)的概率非常大,针对这样的现实情况,本文尝试对现有的出租车调度系统进行改进使其支持跨链的交易操作,设计并实现了跨区域交易的相关测试以及跨区域交易的合约,最终给出了一个简易的能够实现跨链交易的调度过程,设计实验验证了实现的正确性。

  % 关键词用“英文逗号”分隔,输出时会自动处理为正确的分隔符
  \thusetup{
    keywords = {区块链, 车联网, 树状区块链跨子链},
  }
\end{abstract}

\begin{abstract*}
    With the rapid development and widespread application of technology in the automotive industry, the interaction between vehicles will become essential. In order to create a safer and more efficient transportation environment, it is proposed to apply blockchain technology to the Internet of Vehicles. The decentralization and independence of blockchain technology can eliminate intermediaries in taxi dispatch systems. This allows for direct communication and transactions between passengers and drivers, and the use of blockchain technology in online taxi services can help to strengthen relationships between all stakeholders involved.
    
    However, traditional blockchain technology is characterized by a single chain structure, which results in a long chain and consumes significant time and energy when processing a large number of transactions. In this situation, it has poor execution efficiency. To address this issue, we implement a "tree-structured blockchain" in the taxi dispatch system. The tree-structured blockchain divides the sub-chains based on Geohash encoding of geographical locations, forming branching blocks to manage the sub-chains. This creates a tree-structured trie structure in which different sub-chains manage different areas.
    
    This thesis begins by reading some of the source code for tree-structured blockchains and providing annotations for certain core functions. Following this, the article reproduces the use of tree-structured blockchains in a taxi dispatch system based on the existing project structure, and verifies the load performance of parallel running of multiple sub-chains. In addition, the article verifies the correctness of cross-sub-chain asset transfer in tree-structured blockchains and tests its performance. Building on this work, the article attempts to improve existing taxi dispatch systems by enabling cross-regional (cross-sub-chain) transactions. Current taxi dispatch systems only support transactions within a single sub-chain, but in real-world scenarios, the likelihood of passengers and vehicles being located in different areas (different sub-chains) is high. To address this, the author attempts to enable cross-chain transactions in existing taxi dispatch systems by designing and implementing tests and contracts for cross-regional transactions, ultimately providing a simple process for enabling cross-chain transactions and verifying its correctness through experimentation.

  % Use comma as separator when inputting
  \thusetup{
    keywords* = {Blockchain, Internet of Vehicles, Tree-structured Blockchain Cross-Subchain},
  }
\end{abstract*}
