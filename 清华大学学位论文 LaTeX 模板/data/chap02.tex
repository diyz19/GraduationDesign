% !TeX root = ../thuthesis-example.tex

\chapter{go-ethereum的部分源码解读}

\section{整体结构}

go-ethereum的代码结构非常清晰,整个代码库主要分为以下几个部分:

\begin{itemize}
    \item accounts:这是管理以太坊账户的代码,包括账户管理、加密和解密、签名和验证等。
    \item cmd: 包含所有go-ethereum的命令行工具,如geth、abigen、bootnode等。
    \item core: 核心代码,包括区块链数据结构、区块链的实现、账户管理、交易执行等。
    \item crypto: 加密相关的代码,如私钥生成、签名等。
    \item consensus:这是区块链共识算法的实现,包括PoW(Proof of Work)、PoA、Ethash等。
    \item eth:以太坊网络协议的实现,包括区块同步、交易广播、状态传播和客户端协议等。
    \item internal:包含整个项目中使用的内部包。
    \item miner: 包含了以太坊矿工相关的代码,如挖矿、打包交易、广播区块等。
    \item node: 包含了以太坊节点相关的代码,如节点的启动、关闭、管理等。
    \item p2p: 网络层的实现,用于节点之间的通信。
    \item params: 包含了以太坊的参数配置,如区块链难度、网络ID、区块奖励等。
    \item storage:包含了以太坊的存储实现,如LevelDB等。
    \item rpc: 实现了以太坊的JSON-RPC API、WebSocket和IPC等。
    \item whisper: 实现了以太坊的whisper协议,用于点对点的消息传递,实现安全、私密的通信。
    \item trie: 包含了以太坊中使用的Merkle Patricia Trie数据结构相关的代码,如节点的添加、删除、查找等。
\end{itemize}

\section{部分核心源码的功能介绍}

由于笔者毕设的实验目标与跨链与转账有关,故笔者主要重点研究了树状区块链的blockchain结构,以及转账操作中涉及的代码内容。代码说明仓库地址。

\subsection{针对区域索引的代码说明}

首先是对创世块增加position的地理位置属性,创世块写入数据库并存储,实现存储创世块中的账户位置。

其次,对于account账户添加position的地理位置属性,同时实现获取账户位置的函数接口GetPosition,根据账户位置的hash值添加账户位置树;对于transaction交易添加position的地理位置属性。

实现并添加区域状态树,添加区域状态数据库,缓存区域状态,在miner中添加区域状态信息。

\subsection{针对树状多链的代码说明}

区块链将节点大致分为了两类,分别为结构性节点和普通节点,其中,结构性节点的作用便是为了构建并维护树状区块链的树状层次结构,包括根节点、分支节点和叶节点;其余节点均为普通节点。

落实到代码中,首先是实现区块链的树状属性结构,节点中记录分支区块,并增加同步分支区块的方式,分支节点可以生成分支区块,分支区块写入区块链后,会根据regionid同步区块。

\subsection{针对资产转移的代码说明}

在分支节点中管理着资产转移交易列表,该列表结构中存储着子链发起的资产转移交易内容,此外对于添加进该列表的交易,代码还实现了交易的原子性操作。

资产转移交易被定义为特殊的交易类型,其设定为不需要消耗gas,此外,相较于传统交易,资产转移交易额外增加了一个自定义的txtype属性,分支节点在接收到资产转移交易时,根据该属性值的不同进行后续的不同处理。

\section{本章小结}

本章对现有系统使用的树状区块链源码中的部分功能所涉及到的源码作了注释,详细说明在仓库中。

主要作用是方便后续工作者可以更好的理清代码逻辑,进而得以对于整个系统的运作流程更为清晰。同时,也便于未来可能的对现有源码的底层的改进工作的进行。
